\documentclass[sigconf]{acmart}

\usepackage{graphicx}
\usepackage{hyperref}
\usepackage{todonotes}

\usepackage{endfloat}
\renewcommand{\efloatseparator}{\mbox{}} % no new page between figures

\usepackage{booktabs} % For formal tables

\settopmatter{printacmref=false} % Removes citation information below abstract
\renewcommand\footnotetextcopyrightpermission[1]{} % removes footnote with conference information in first column
\pagestyle{plain} % removes running headers

\newcommand{\TODO}[1]{\todo[inline]{#1}}

\begin{document}
\title{Gerrymandering Detection Using Data Analysis}

\author{Kevin Duffy}
\orcid{1234-5678-9012}
\affiliation{%
  \institution{Indiana University}
  \streetaddress{4014 E. Stop 10 Rd.}
  \city{Indianapolis} 
  \state{Indiana} 
  \postcode{46237}
}
\email{kevduffy@iu.edu}

% The default list of authors is too long for headers}
\renewcommand{\shortauthors}{K. Duffy}

\begin{abstract}
Can the evergreen issue of partisan gerrymandering be solved using data and algorithms? That question is closer to being solved than ever, as a method developed by University of Chicago researchers in 2014 is currently pending approval by the United States Supreme Court. We examine the method, known as the efficiency gap model, using data from the 2012 congressional elections. We determine the most gerrymandered states by creating a model that applies the efficiency gap method to this dataset. We then evaluate the effectiveness of this model and take stock of any substantial critiques and alternative methods.
\end{abstract}

\keywords{Big Data, Elections, Gerrymandering, Voting, Efficiency gap}

\maketitle

\section{Introduction}
In 1964 the Supreme Court established in constitutional law the principle of "one-person, one-vote".\cite{oneperson} The idea appears self-evident; the value of any citizen's vote is equal to that of any other citizen's vote. But could anything still exist in our institutions of our democracy that resists this principle? Beyond obvious impediments to voting such as the since-repealed prohibition on women or racial minorities voting, what other barriers could exist? And why has "one-person, one-vote" become an issue as important, and contentious, as ever?

The barrier, many will argue \cite{wapo}\cite{chicago}\cite{thornburg}, lies in the concept of "gerrymandering", or the manipulation of legislative district lines for the benefit of one political player over another. But determining whether something is gerrymandered has proven to be a difficult task. And even once you decide something is gerrymandered, what can be done about it? Answers to these questions may be coming in the form of both advanced data analysis, and simple arithmetic.

In this paper we will lay the foundation of the issue: what is gerrymandering, when has it been used successfully, and what have people done to attempt to curb it? We will then explore possible tests to identify gerrymandering, particularly the efficiency gap method which is currently be litigated before the United States Supreme Court. We will replicate the efficiency gap method using congressional data to determine the greatest and least gerrymandered states, according to the model and suggest avenues for further study. We will take a cursory look at other detection methods that have been proposed. We will examine critiques of anti-gerrymandering efforts, including detection models, redistricting solutions, and the idea that gerrymandering should be regulated whatsoever. We will then conclude with an examination of proposals to solve gerrymandering after detection is completed, and the role data applications are playing in that study.

\section{What is gerrymandering?}
The building blocks of America's democratic republic are legislative districts. Any one American lives in three overlapping districts: their State Senate district, their State House of Representatives district, and their U.S. House of Representatives (congressional) district. There is a set limit of 435 congressional districts dispersed to the 50 states based on population measured at the last U.S. Census. Within each state, each congressional district must be exactly the same size. Across the nation as a whole, the average population of a district is 710,000 residents. State House and Senate districts are given a 10 percent population range, as long as it is not systematically used to give one party an advantage. The federal government leaves it to the states to draw the lines all three of these maps. While there is some variance, most states have these lines drawn by the state legislature themselves. The lines are redrawn every ten years, after the census is conducted, in a process called redistricting.\cite{maps}\cite{population}\cite{govtrack}

Gerrymandering, simply put, is the process by which a political party in power uses redistricting to "manipulate district boundaries to create maps that systematically advantage the party in control and lock in an advantage for the party in future elections", according to the NYU Brennan Center of Justice.\cite{brennan} In other words, the political party seeks to maximize the efficiency of each vote for their candidates, while decreasing the efficiency of the opposite party. This is done primarily through by "cracking" and "packing" the district maps:
\begin{itemize}
    \item \textit{Cracking} means splitting up a bloc of voters loyal to one party into several other districts, thus diluting the power of their collective vote and maximizing the amount of districts the preferred party is competitive in.
    \item \textit{Packing} means concentrating a bloc of voters loyal to the opposing party into one district, giving them an overwhelming share of the vote in that district, but decreasing 
    their power in several other districts. \cite{chicago}
\end{itemize}
The overall effect of cracking and packing is maximize the "wasted votes" of the disfavored party, while minimizing those of the favored party. The concept of wasted votes is crucial to the model we will examine further. A wasted vote can be defined as one of the following:
\begin{itemize}
    \item For the winning candidate (party A), a wasted vote is any vote beyond the threshold needed to win, or 50 percent of the vote. 
    \item Every vote for the losing candidate (party B) is considered wasted, as the vote did not net a seat for that party.\cite{chicago}
\end{itemize}
You can clearly see the potential for cracking and packing to greatly effect the amount of wasted votes a certain party receives. When a party's voters are cracked into several different districts, the net effect is that their votes go to various losing candidates, thus the votes are all wasted. When a party's voters are packed into one concentrated district, the net effect is that their votes overwhelmingly elect just one candidate, well beyond the needed threshold to win. Thus, all the votes beyond that threshold are wasted.

\subsection{History}
This is not merely a theoretical exercise. Gerrymandering has been utilized throughout the history of the United States by virtually every political party that has been in power. In fact, the term "gerrymander" dates back to 1812, when the \textit{Boston Gazette} used the phrase to decry a unfairly-drawn redistricting plan signed into law by Massachusetts Governor Elbridge Gerry.\cite{griffith} A famous political cartoon depicts one particularly contrived looking district as a dragon-esque creature, while others compared its shape to that of a salamander. The colloquial term became "gerry-mander" after the governor who enabled such a result.

The utilization and effectiveness of gerrymandering does not appear to have lessened with time. One of the most effective uses of the practice happened earlier this decade. 

In 2008, the national Republican Party was in a dire position. Barack Obama had just been elected president in a sweep that included control of both houses of Congress. They held the House of Representatives by the largest margin seen in almost 20 years.\cite{house} Journalist Michael Grunwald presented a grim narrative for the party in Time magazine in 2009, writing that "polls suggest that only one-fourth of the electorate considers itself Republican, that independents are trending Democratic and that as few as five states have solid Republican pluralities." In addition, he pointed out that the overall population was decreasing in demographics that had proven to be solidly Republican - "less white, less rural, less Christian".\cite{distress}

Fast-forward to 2017: Republicans control the White House and Congress, owning the House by almost as large a margin as the Democrats did just 8 years earlier.\cite{house} What happened to produce results that so starkly contrast with the outlook Grunwald predicted?

\subsection{REDMAP}

In the wake of the 2008 election, the Republican State Leadership Committee launched the Redistricting Majority Project (REDMAP).\cite{redmap} The strategy of the plan was brilliant in its simplicity: use their funding to target state legislature races in order to control as many state legislatures as possible when the next redistricting occurred in 2011. In most states, the task of drawing new district boundaries is left to the state legislatures.\cite{maps} Redistricting occurs after each census to reflect changes in population densities and demographics. The strategy paid off in giving Republicans control of the redistricting process in many of the states. From there, partisan politics began its work.

REDMAP was a clear success as evidenced by the ensuing 2012 election, the year in which Barack Obama won reelection. In Michigan, for example, voters cast 240,000 more votes overall for Democrats in congressional races, but 9 Republicans were elected and only 5 Democrats. In Pennsylvania, voters cast 83,000 more votes overall for Democrats in congressional races, but 13 Republicans were elected and only 5 Democrats. Across the nation, Republicans won 54 percent of house seats and 58 out of 99 state legislative chambers, while winning only 8 out of 33 Senate races (which are gerrymander-proof as they do not rely on district lines).\cite{redmap}

The effect of gerrymandering is evident in the results of REDMAP. Republicans maximized the wasted votes of the Democrats and minimized the wasted votes of the Republicans. Although in many cases Democrats had more votes statewide, more Republicans candidates were sent to Congress. 

Gerrymandering is an issue seen by many in both parties as problematic. Representative Brian Fitzpatrick (R-PA) wrote in \textit{The Hill} that gerrymandering has caused the nation to stray from its ideal of representative leadership, as it has "has undermined community-focused representation by forcing lawmakers to ideological extremes and exacerbating electoral complacency that causes lawmakers to focus on accumulating power rather than serving constituents."\cite{bipartisan}

However, despite bipartisan efforts in the legislature such as those lauded by Fitzpatrick, the most promising avenue for curbing gerrymandering may lie in a different branch of government: the United States Supreme Court. 

\subsection{The Supreme Court and Gerrymandering}
The court has already banned racial gerrymandering in the decision on \textit{Thornburg v. Gingles} in 1986. They determined that a district map in North Carolina violated the Voting Rights Act by gerrymandering the districts in a way that unfairly dilluted the power of black voters.\cite{thornburg}

But up until recently, partisan gerrymandering has been left to the states to police themselves. The Supreme Court has heard around 50 cases in its history imploring the court to intervene against a partisan gerrymandered map, and each time has deferred.\cite{chicago} The last major case was \textit{Vieth v. Jubelirer} in 2004. Justice Antonin Scalia, since deceased, delivered the majority opinion stating that the courts were not responsible for partisan gerrymandered maps as they were "non-justiciable".\cite{wapo}

Justice Anthony Kennedy is reliable in his unreliability - he serves as the court's "swing vote", as court observers are often unsure which way he'll fall on a given issue until the decision is handed down. Given the court's ideological polarity often leads to close votes, this arguably makes him the most powerful justice on the bench. In \textit{Vieth}, Kennedy voted along with the majority opinion that upheld the allegedly-gerrymandered maps. However, he left open the possibility of the court adjudicating gerrymanders, if a clear standard could be found for determining whether a map is gerrymandered or not.\cite{wapo}

Such a standard has not been apparent until, perhaps, now.

\section{The Efficiency Gap}
In 2017, a case reached the Supreme Court alleging that the Wisconsin State Assembly was gerrymandered in such an extremely partisan way as to render it unconstitutional. The plaintiffs, savvy enough to recognize Justice Kennedy as the potential swing vote and remembering his desire for a clear standard, argued their case using the "efficiency gap" method.\cite{wapo}

The efficiency gap method was developed by University of Chicago law professor Nicholas Stephanopoulos and Eric McGhee, a researcher at the Public Policy Institute of California.\cite{chicago}

The efficiency gap is a relatively simple formula, based on the aforementioned concept of wasted votes. The formula takes the total statewide wasted votes of party A and subtracts the total statewide wasted votes of party B, and then divides that number by the total number of statewide votes to find the efficiency gap score.\cite{chicagoformula}

\[Gap=\frac{Waste(A)-Waste(B)}{TotalVotes}\]

Another way it can be written:

\begin{center}
    Gap=(Seat margin) - (2 x Vote margin)
\end{center}

The "seat margin" is the percentage of seats you win from the statewide allotment minus 50 percent, and the "vote margin" is the total percentage of the vote you win minus 50 percent. A negative result means the map is biased against you.\cite{chicagoformula} This is a helpful format when we begin measuring the net effect of gerrymandering in congressional districts.

If the two parties have similar numbers of wasted votes, or neither party has a significant amount of wasted votes, the efficiency gap score for that state will be low indicating acceptable levels of map bias. However, if one party has a disproportionate number of wasted votes compared to its opponent, the result will be a higher efficiency gap score indicating unacceptable levels of map bias.

This method captures the effects of both cracking and packing: packing will be detected by the wasted votes from an excessive victory, and cracking will be detected excessive amounts of losing votes statewide.

Let's apply this to an example. Let's say Party A and Party B are competing in a state with ten congressional districts of 100 people each.

\begin{center}
\begin{tabular}{ |c|c|c| } 
 \hline
 District & Party A votes & Party B votes \\
 \hline
 01 & \textbf{90} & 10 \\
 \hline
 02 & 49 & \textbf{51} \\
 \hline
 03 & 45 & \textbf{55} \\
 \hline
 04 & \textbf{95} & 5 \\
 \hline
 05 & 45 & \textbf{55} \\
 \hline
 06 & \textbf{90} & 10 \\
 \hline
 07 & 49 & \textbf{51} \\
 \hline
 08 & 45 & \textbf{55} \\
 \hline
 09 & \textbf{95} & 5 \\
 \hline
 10 & 45 & \textbf{55} \\
 \hline
\end{tabular}
\end{center}

In Districts 01, 04, 06, and 09, Party A wins by an overwhelming margin. In the rest of the districts, Party B wins by narrow margins. This results in more votes being cast for Party A statewide, but Party B gets more seats:

\begin{center}
    \begin{tabular}{|c|c|c|}
    \hline
            & Party A & Party B \\
    \hline
      Votes & 648 & 352 \\
    \hline
      Seats & 4 & 6 \\
    \hline
    \end{tabular}
\end{center}

These races result in an overwhelming amount of wasted votes for Party A, and a minimal amount for Party B.

\begin{center}
\begin{tabular}{ |c|c|c|c| } 
 \hline
 District & Party A Wasted votes & Party B Wasted votes \\
 \hline
 01 & 40 & 10 \\
 \hline
 02 & 49 & 1 \\
 \hline
 03 & 45 & 5 \\
 \hline
 04 & 45 & 5 \\
 \hline
 05 & 45 & 5 \\
 \hline
 01 & 40 & 10 \\
 \hline
 02 & 49 & 1 \\
 \hline
 03 & 45 & 5 \\
 \hline
 04 & 45 & 5 \\
 \hline
 05 & 45 & 5 \\
 \hline
 \textbf{Total} & \textbf{448} & \textbf{52} \\
 \hline
\end{tabular}
\end{center}

We can see that Party A has many more wasted votes than Party B, indicating the map may be drawn to minimize the efficiency of Party A. We then add up the total number of votes cast statewide and plug these numbers into our efficiency gap formula:

\begin{displaymath}
\frac{448-52}{1000} = \frac{396}{1000} = 0.396\
\end{displaymath}


So our efficiency gap, written as a percentage, is 39.6 percent. The map is clearly tilted in favor of Party B. But is it considered illegal gerrymandering? In their paper, the authors establish thresholds for when an efficiency gap indicates levels of illegal gerrymandering:

\begin{itemize}
    \item For state legislature maps, an efficiency gap score above eight percent is considered illegally gerrymandered. The mere percentage is used as each legislature is an entity unto itself, elected wholly by voters in the state. This along with variances in size among state legislatures, makes efficiency gap the best way to normalize disparate state houses for comparison.
    \item For congressional maps, a state is considered illegally gerrymandered if the map costs a party two seats. In contrast to state houses, the authors contend, "aggregate House seats are the parties' main objective". In that regard, seats are the best way to normalize disparate state sizes for comparison.\cite{chicagothreshold}
\end{itemize}

If we write our formula in the format (Seat margin) - (2 x Vote margin), we can measure how many seats were lost as a result of the biased map. In this example, Party A won 64.8 percent of the vote, but was awarded only 4 out of the 10 seats. For Party A:

\begin{center}
    (.40-.50) - (2 x .148)
    
    -.10 - .296
    
    -.396
\end{center}

Now let's give Party A enough seats to make the efficiency gap score as close to 0 as possible. We will say that Party A in this alternate scenario received 8 seats, represented as .80 in the seat share value:

\begin{center}
    (.80-.50) - (2 x .148)
    
    .30 - .296
    
    .004
\end{center}

We have brought the score effectively to 0. So using this formula, we have determined that the efficiency gap derived from the biased map cost Party A a total of 4 seats, well above the threshold for illegal gerrymandering.

\section{Application}

We implemented the efficiency gap method into a python application powered by real-world election data in order to determine whether the district maps for Indiana's House of Representatives and State Senate pass or fail the efficiency gap thresholds. 

Indiana was chosen arbitrarily, primarily because it is the home to both the author and institution of this paper. In addition, Indiana's legislature gives us clean and uniform data to work with, as there are an even number of senators and representatives elected, no special elections, and no run-off elections to complicate the data. However, after implementation of the application, it became apparent that it was fortunate Indiana was chosen as the results showcase important teaching moments in understanding the efficiency gap and its applications.

Because we were evaluating state legislatures, we did not have to calculate seats lost, so the results are given in raw efficiency gap score. 

\subsection{Data sourcing and cleaning}

For our data, we use the election results from the 2016 Indiana House of Representatives races\cite{houseresults}, and the 2014 and 2016 Indiana State Senate races\cite{senateresults2014}\cite{senateresults2016}. The data was collected from Ballotpedia, an online election and candidate encyclopedia. For context, each member of the House is elected every even year for a two-year term. There are 100 representatives. Each member of the Senate is elected to a four-year term, with elections occurring every two years to elect half of the members. There are 50 senators. Thus, to receive a full sample of the House races, we only needed to collect data from one election year. But for the Senate, we needed two election years in order to collect data for the full senate.

There were two complicating factors with the data that needed to be cleaned before implementation into our model:
\begin{enumerate}
    \item One third of the races in 2016 were uncontested, meaning the winning candidate had no opposition to compare to. In 2014, almost half were uncontested. Depending on the county data recording, these are represented in one of two ways. 
        \begin{enumerate}
            \item The votes cast for the winner are displayed, resulting in an election that looks like 20,000 votes were cast for candidate A, and 0 votes were cast for candidate B.
            \item No votes are displayed, and the winner is simply displayed as a default.
        \end{enumerate}
    Both of these taken at face value are problematic for our model. In the first case, plugging these results into our model could overstate how many wasted votes there were for the winning candidate, as the model would think that the winner received 20,000 more votes than they needed to, and the loser received no wasted votes. This is unlikely to occur in reality if the opposing party had fielded a candidate. Even if it is not a close race, the loser would accumulate enough votes to alleviate the amount of wasted votes accrued by the winner and increase the amount of wasted votes for the loser. 
    
    In the second case, rather than overstating the wasted votes, they are understated. The race is treated as a draw in terms of wasted votes, when uncontested elections would in reality be a major symptom of an efficiency gap and wasted votes should be accrued. 
    
    Clearly, they cannot be ignored. The efficiency gap authors provide guidance on what to do with these races. For state house races, they ran a multi-level model using a fixed effect for incumbency and random effects for year, state, and district. If the district had been contested in its past, the value was derived from other districts in the state during that year along with the same district in other years. If not, they had a random draw of random effects. \cite{chicagouncontested}
    
    The results were a mean Democratic vote share of 66 percent for uncontested Democratic candidates, with 90 percent of values falling between 52 and 83 percent. Democratic vote share for races with uncontested Republicans was placed at 36 percent, with 90 percent of values falling between 22 and 43 percent. The authors do not hold this solution to be the be-all-end-all model for computing vote shares of uncontested candidates, as they "encourage scholars to explore a range of imputation techniques."\cite{chicagouncontested}
    
    Our solution was to uncritically use the authors' figures of 34 percent share for Republicans in uncontested Democratic seats, and 36 percent share for Democrats in uncontested Republican seats. For those seats that had no winning vote data available, we took the average population of a district, adjusted for that year's vote turnout, and applied the percentages to that number. For those seats with winning vote data, we simply took half of the winning votes as the loser's share of votes.
    
    If further work to be undergone on this application, we would recommend fine tuning these calculations, particularly if one were to specifically focus on a particular state legislature, as vote shares for a given political party would most likely vary from state to state.
    \item The other complicating factor for this experiment was the existence of third parties. In the efficiency gap calculations, third party votes are ignored, relying on the two-party vote.\cite{gill} For most districts, the effect that third parties have is marginal:
    \begin{itemize}
        \item The United States is a two-party system, mostly due its "winner take all" election rules (where the party with the most votes is the singular winner in a given race, whereas a proportional system would give distribute legislative seats proportionally based on vote share). Third parties therefore have a difficult time gaining any sort of power:
        \begin{itemize}
            \item The highest vote share of any third party in the 2016 presidential election was 4 percent for the Libertarian Party, the highest share the Libertarian Party had ever received in a presidential election.\cite{libertarian}
            \item There are no third party members of the Indiana House of Representatives and the Indiana State Senate.\cite{senateresults2016}\cite{houseresults}
            \item There are no third party members of the U.S. House of Representatives.\cite{house}
        \end{itemize}
        \item Third parties are varied; there is no one singular third party to claim a stake in the redistricting process. Thus, their voice is diluted by diversification. 
        \item When we establish that we are operating under a binary party system, third parties make no difference in the efficiency gap formula, as a vote cast for a third party candidate is wasted for Democrats and Republicans equally, thus cancelling itself out.
    \end{itemize}
    The solution was straightforward - we simply removed third party votes from our calculations and operated under a two-party vote system.
\end{enumerate}

\subsection{Implementation}

The application made moderate use of the Python Pandas module. We began by importing two dataframes: the 2016 House results, and the 2014 and 2016 Senate results combined into one dataframe. Because the efficiency gap is a simple formula, the values needed are similarly simple. The only values needed were the Republican votes and Democratic votes for each district:

\begin{center}
\begin{tabular}{ |c|c|c| }
    district & dvotes & rvotes \\
    1 & 15561 & 7780 \\
    2 & 24820 & 12786 \\
\end{tabular}
\end{center}
\begin{center}
...
\end{center}
\begin{center}
\begin{tabular}{ |c|c|c| }
    99 & 24820 & 12786 \\
    100 & 14110 & 7055 \\
\end{tabular}
\end{center}

With the data imported, the first step is to calculate the wasted votes for both parties. We have two separate functions to calculate Democratic wasted votes and Republican wasted votes.
\begin{quote}
def dwaste(row):
    
\indent if row['dvotes']\textgreater row['rvotes']:
    
\indent \indent val = row['dvotes]-((row['dvotes']+row['rvotes'])*.5)
    
\indent else:
    
\indent \indent val = row['dvotes']
    
\indent return val
\end{quote}

The Republican wasted votes function is identical except the 'dvotes' and 'rvotes' values are switched. This portion of the script goes line by line through the dataframe to calculate the wasted votes for each party per district. This needs to be done row by row as wasted votes cannot be found as an aggregate statewide total, but by looking at each individual district race. 

Next, we applied the rwaste() and dwaste() functions to our data frames, and then we can get our statewide totals of wasted votes by party:

\begin{quote}
    df['rwaste'] = df.apply(rwaste, axis=1)
    
    rtotal = df['rwaste'].sum()
\end{quote}

From there, we plugged the statewide wasted votes totals into our efficiency gap formula:

\begin{quote}
((dtotal - rtotal) / ((df['rvotes'].sum() + df['dvotes'].sum())) * 100
\end{quote}

Finally, we implemented a simple function that serves as our threshold test. If the formula falls above 8, it triggers an "UNCONSTITUTIONAL GERRYMANDER" response:

\begin{quote}
    def eg():
    \indent if final > 8
    \indent \indent print("UNCONSTITUTIONAL GERRYMANDER")
    \indent else:
    \indent \indent print("ACCEPTABLE")
\end{quote}

Since the data used for this particular experiment is small, the application was able to be executed on a personal computer using an Ubuntu virtualbox. 

\subsection{Results}

The application revealed that the House of Representatives, with 2,992,624 votes cast, had 838,675 Democratic wasted votes and 657,637 Republican wasted votes resulting in an efficiency gap score of 6.05 percent in favor of Republicans. This falls under the 8 percent threshold, indicating that if the efficiency gap were to be adopted as a court standard by the Supreme Court, this map would be ruled constitutional.

On the other hand, the Senate, with 2,107,263 votes cast, has 661,509 Democratic wasted votes and 347,122 Republican wasted votes resulting in an efficiency gap score of 15.58 percent in favor of Republicans. This lands well above the gerrymandering threshold. If this standard were to be adopted by the Supreme Court, there is a decent chance the Senate map would be ruled unconstitutional. 

There are a few factors to consider that may be used to explain the discrepancy with the Senate vote:
\begin{itemize}
    \item If we take at face value that the Senate is twice as gerrymandered as the House, a major reason could be district size. The House has twice as many districts as the Senate over the same land area. The more granular a district is, the more difficult it becomes for a map to be gerrymandered, as you have smaller populations and smaller land areas per district. People of similar ideologies and political leanings tend to group together, so with smaller parameters, it becomes more difficult to group some of these individuals with opposing parties in order to "crack" their vote. 
    \item If we are skeptical, the results could be explained by the fact that half of the Senate data was taken from the 2014 midterm election, while all of the House data was taken from the 2016 presidential election. 
    \begin{itemize}
        \item Midterm elections tend to have lower voter turnout, so the data may not be as accurately reflect the true political landscape of a region. In 2014, Indiana had a voter turnout rate of 28 percent, the lowest in the nation that year\cite{2014turnout}, compared to 58 percent in 2016.\cite{2016turnout}
        \item Historically, Republicans tend to have higher turnout in midterm elections. Nationally, Republicans were 20 percent more likely to vote in 2010 and 2014 than Democrats were, according to an analysis by the New York Times' Nate Cohn. \cite{cohn}
    \end{itemize}
    Further analysis would need to be done in order to determine if this explanation suffices. It would, in fact, be in the Republicans best interest to find an alternative explanation other than gerrymandering. The efficiency gap authors, in outlining their proposed court test, allow that states above the threshold could show that the the gap was the result of either the "consistent application of legitimate policies", or "inevitable due to the states' underlying political geography."\cite{chicagothreshold}
\end{itemize}

The question remains whether this standard will be used to measure map bias and judge gerrymandering. During oral arguments for \textit{Whitford} in October 2017, Chief Justice John Roberts referred to the theory as "sociological gobbledygook".\cite{gilltranscript} But some court observers are anticipating Justice Kennedy, the swing vote, to vote in favor of the efficiency gap test.\cite{analysis}

\section{Criticism}
\subsection{Efficiency gap shortfalls}
\subsection{Philosophical differences}

\section{Alternative detection methods}

\section{Gerrymandering Solutions}

\section{Conclusion}


\begin{acks}

  The author would like to thank Dr. Gregor von Laszewski and all TA's for their tireless work in ensuring this class goes smoothly.

\end{acks}

\bibliographystyle{ACM-Reference-Format}
\bibliography{gerrymander} 

\appendix

\section{Issues}

\DONE{Example of done item: Once you fix an item, change TODO to DONE}

\subsection{Assignment Submission Issues}

    \DONE{Do not make changes to your paper during grading, when your repository should be frozen.}

\subsection{Uncaught Bibliography Errors}

    \DONE{Missing bibliography file generated by JabRef}
    \DONE{Bibtex labels cannot have any spaces, \_ or \& in it}
    \DONE{Citations in text showing as [?]: this means either your report.bib is not up-to-date or there is a spelling error in the label of the item you want to cite, either in report.bib or in report.tex}

\subsection{Formatting}

    \DONE{Incorrect number of keywords or HID and i523 not included in the keywords}
    \DONE{Other formatting issues}

\subsection{Writing Errors}

    \DONE{Errors in title, e.g. capitalization}
    \DONE{Spelling errors}
    \DONE{Are you using {\em a} and {\em the} properly?}
    \DONE{Do not use phrases such as {\em shown in the Figure below}. Instead, use {\em as shown in Figure 3}, when referring to the 3rd figure}
    \DONE{Do not use the word {\em I} instead use {\em we} even if you are the sole author}
    \DONE{Do not use the phrase {\em In this paper/report we show} instead use {\em We show}. It is not important if this is a paper or a report and does not need to be mentioned}
    \DONE{If you want to say {\em and} do not use {\em \&} but use the word {\em and}}
    \DONE{Use a space after . , : }
    \DONE{When using a section command, the section title is not written in all-caps as format does this for you}\begin{verbatim}\section{Introduction} and NOT \section{INTRODUCTION} \end{verbatim}

\subsection{Citation Issues and Plagiarism}

    \DONE{It is your responsibility to make sure no plagiarism occurs. The instructions and resources were given in the class}
    \DONE{Claims made without citations provided}
    \DONE{Need to paraphrase long quotations (whole sentences or longer)}
    \DONE{Need to quote directly cited material}

\subsection{Character Errors}

    \DONE{Erroneous use of quotation marks, i.e. use ``quotes'' , instead of " "}
    \DONE{To emphasize a word, use {\em emphasize} and not ``quote''}
    \DONE{When using the characters \& \# \% \_  put a backslash before them so that they show up correctly}
    \DONE{Pasting and copying from the Web often results in non-ASCII characters to be used in your text, please remove them and replace accordingly. This is the case for quotes, dashes and all the other special characters.}
    \DONE{If you see a figure and not a figure in text you copied from a text that has the fi combined as a single character}

\subsection{Structural Issues}

    \DONE{Acknowledgement section missing}
    \DONE{Incorrect README file}
    \DONE{In case of a class and if you do a multi-author paper, you need to add an appendix describing who did what in the paper}
    \DONE{The paper has less than 2 pages of text, i.e. excluding images, tables and figures}
    \DONE{The paper has more than 6 pages of text, i.e. excluding images, tables and figures}
    \DONE{Do not artificially inflate your paper if you are below the page limit}

\subsection{Details about the Figures and Tables}

    \DONE{Capitalization errors in referring to captions, e.g. Figure 1, Table 2}
    \DONE{Do use {\em label} and {\em ref} to automatically create figure numbers}
    \DONE{Wrong placement of figure caption. They should be on the bottom of the figure}
    \DONE{Wrong placement of table caption. They should be on the top of the table}
    \DONE{Images submitted incorrectly. They should be in native format, e.g. .graffle, .pptx, .png, .jpg}
    \DONE{Do not submit eps images. Instead, convert them to PDF}

    \DONE{The image files must be in a single directory named "images"}
    \DONE{In case there is a powerpoint in the submission, the image must be exported as PDF}
    \DONE{Make the figures large enough so we can read the details. If needed make the figure over two columns}
    \DONE{Do not worry about the figure placement if they are at a different location than you think. Figures are allowed to float. For this class, you should place all figures at the end of the report.}
    \DONE{In case you copied a figure from another paper you need to ask for copyright permission. In case of a class paper, you must include a reference to the original in the caption}
    \DONE{Remove any figure that is not referred to explicitly in the text (As shown in Figure ..)}
    \DONE{Do not use textwidth as a parameter for includegraphics}
    \DONE{Figures should be reasonably sized and often you just need to
  add columnwidth} e.g. \begin{verbatim}/includegraphics[width=\columnwidth]{images/myimage.pdf}\end{verbatim}

re


\end{document}
